\paragraph{Experiments preparation and data collection}

% Cenni al experiment_handler

\subsection{Mathematical description}

Il modello fisico può essere modellizzato in equazioni differenziali ordinarie (EDOs) suddividendo il sistema in singole componenti: motore DC, trasmissione, molla, massa, molla, massa. I parametri utilizzati sono i valori nominali forniti nei manuali di setup delle componenti.

% \includegraphics{model}

% Equazioni componenti
\[	V = R_m i + L_m \frac{d}{dt}i + k_m \dot{\theta}_m \]
\[ \tau_m = \eta_m k_t i , \quad \tau_{ml} = \eta_g K_g \tau_m = \eta_m \eta_g K_g k_t i \]
\[ J_m \ddot{\theta}_l + B_m \dot{\theta}_l = \tau_{ml} - K_{s1} ( \theta_l - \theta_1 ) \]
\textit{Model of the 1-d.o.f. system}
\[ J_1 \ddot{\theta}_1 + B_1 \dot{\theta}_1 = K_{s_1} ( \theta_l - \theta_1 ) \]
\textit{Model of the 2-d.o.f. system}
\\
\[ J_1 \ddot{\theta}_1 + B_1 \dot{\theta}_1 = K_{s_1} ( \theta_l - \theta_1 ) - K_{s_2} ( \theta_1 - \theta_2 ) \]
\[ J_2 \ddot{\theta}_2 + B_2 \dot{\theta}_2 = K_{s_2} ( \theta_1 - \theta_2 ) \]

Questa formulazione del problema è utile per il progetto del regolatore, che baseremo su una rappresentazione in spazio di stato ed una funzione di trasferimento.

% Forma di stato
% TF

% Immagine simulink: componenti separate

Per poter verificare la correttezza di questo modello, si sono condotti degli esperimenti sul sistema reale, innanzitutto collegando solo la prima massa, poi entrambe.
Imponendo una tensione sul motore, si è confrontata la risposta in anello aperto del sistema fisico con la simulazione eseguita da Simulink e basata sul modello matematico.


Si è innanzitutto verificato il guadagno statico della FdT dall'input di controllo del motore alla velocità del primo carico. Imponendo una tensione costante, terminato il transitorio di assestamento della massa, si è osservato che il modello matematico sottostimava la velocità di regime.
Abbiamo manualmente modificato i parametri che più inficiano il guadagno statico: la resistenza rotorica del motore, l'efficienza del motore, l'efficienza meccanica della trasmissione, la costante corrente-coppia e, infine, la costante velocità-tensione (causa della forza elettromotrice).

Per fare questo, si sono condotti esperimenti a velocità differenti (entrambe le direzioni di moto, da velocità molto basse alle massime consentite dai limiti elettrici del motore). Da ciò, si è osservato che il comportamento del sistema non è perfettamente lineare rispetto alla tensione imposta al motore. Alcune interpretazioni fisiche sono state avanzate: l'efficienza meccanica della trasmissione, probabilmente, è sottostimata a basse velocità; per quanto riguarda la $k_m$ (causa della forza elettromotrice), una stima ti tale parametro è difficile dal momento che il suo effetto contrasta la potenza del motore in modo significativo ad alte velocità, ma pressoché nullo a basse velocità.
% parametri modificati ---> G_opt


Questa taratura non permette di migliorare il comportamento del modello matematico durante il transitorio: la sovra-elongazione iniziale e le successive oscillazioni dipendono da poli complessi coniugati, legati ai parametri fisici delle molle e masse (costante elastica, inerzia e coefficiente di frizione). Una modifica manuale di questi valori sarebbe stata particolarmente gravosa, dal momento che si tratta di un approccio trial-and-error non basato su precise deduzioni razionali (il manuale di riferimento delle componenti meccaniche indica che i parametri nominali sono stati ottenuti sperimentalmente).
Una soluzione a questo problema è svincolarsi dal modello puramente matematico (white-box) e sfruttare i dati sperimentali.


\subsection{Black-box identification}

\textit{Black-box identification with a parametric method of TF in time-domain}
\\ \par Un approccio a scatola nera non tiene conto in alcun modo del modello matematico e delle sue le grandezza fisiche; al contrario, si basa esclusivamente sui dati raccolti in laboratorio tramite esperimenti.

Gli esperimenti atti all'identificazione del modello, tutti con tensione del motore come variabile di input, sono di due tipi: scalini di varie ampiezze e sinusoidi a frequenza variabile. 



Dal momento che questo approccio non include un significato fisico, ha senso stimare solo la FdT di nostro interesse: dalla tensione alla velocità delle masse 1 oppure 2, in funzione nel numero di dof. Poiché la TF vede solamente i poli osservabili e controllabili, il numero di poli imposto alla funzione di identificazione è ottenuto osservando i risultati del computer: se aggiungendo ulteriori poli, il risultato coincide con quello di un numero inferiore di poli, allora si è ottenuto l'ordine del sistema.
Nel nostro caso, gli esperimenti ad 1 dof identificano un sistema del 3° ordine, mentre 

\paragraph{Step inputs}

L'intervallo di tensioni applicate varia da -10V a 10V, ad intervalli di 2V (eccetto lo scalino a 0V), per un totale di 10 esperimenti.
I dati raccolti dal sensore di velocità della massa (encoder) sono stati collezionati in un set di esperimenti tramite le funzioni Matlab \textit{iddata()} e \textit{merge()}. Il set di esperimenti così costruito è l'input della funzione \textit{tfest()}: l'output ottenuto è il modello identificato tramite approccio black-box.

Il limite di questo procedimento è che si è totalmente persa la dimensione fisica del problema, compresa una rappresentazione in forma di stato basata sulle variabili fisiche.

\subsection{Gray-box identification}

\textit{Gray-box identification with a parametric error method of state-space time-domain}
\\ \par Il vantaggio di un approccio grey-box sta nel conservare il modello matematico e calibrare i parametri utilizzando i dati sperimentali raccolti. Il modello matematico è basato sui parametri nominali, di cui è definita anche l'incertezza (riportata nella documentazione): in questo modo, l'identificazione può tarare i ciascun parametro in un intervallo di valori ben definito e limitato, ciascuno con un significato fisico. Riguardo i parametri fisici delle masse (momento d'inerzia e coefficiente d'attrito), l'incertezza è fissata a valori tra il 10\% e 20 \%.

Operativamente, abbiamo collezionato i dati di vari esperimenti e eseguito la stima del modello tramite \textit{greyest()}.



